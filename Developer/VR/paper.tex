\NeedsTeXFormat{LaTeX2e}
\documentclass[10pt,a4]{scrartcl}

\usepackage{tabularx,longtable,graphicx,a4,listings,wrapfig,subfigure,textcomp,ccaption,latexsym,setspace,algorithm,algorithmic,esvect,color}
\usepackage[absolute]{textpos}

\ifx\pdfoutput\undefined
  % We're not running pdftex
  % european (better) fonts -- does not look good with pdflatex
  \usepackage[T1]{fontenc}
  \newcommand{\href}[2]{#2\\{\hspace*{5mm}\scriptsize <#1>}\\}
\else
  \pdfcompresslevel=9
  \def\pdfBorderAttrs{/Border [0 0 0] } % No border around Links
  \usepackage{hyperref}
\fi

\title{Project, Virtual Reality}
\author{Stefan Eilemann, Benjamin Long}

\newcommand{\tm}{\texttrademark~}
\newcommand{\rc}{\raise 1ex\hbox{{\tiny\textregistered}}~}
\newcommand{\fig}[1]{Figure~\ref{#1}}
\newcommand{\sref}[1]{Section~\ref{#1}}
\newcommand{\aref}[1]{Appendix~\ref{#1}}
\newcommand{\link}[1]{\htmladdnormallink{#1}{#1}}
\newcommand{\fix}[1]{\textbf{\color{red}{#1}}}
\indentcaption{2em}

% suppress  single floating lines on top (widow) and bottom(club)
%  10000 is infinity
%  tradeoff: maybe underfull vboxes
\clubpenalty=10000
\widowpenalty=10000

\begin{document}
\maketitle

\section{Project Description}

The project builds upon the Equalizer parallel rendering framework and one of
its example programs, eqPly. It adds the following features to either the
core framework, where it is generic, or to the application, where it is not:
\begin{itemize}
\item Render any 3D object in the ply file format on any display system
  supported by Equalizer (LCD, stereo screen, CAVE, display wall) using
  monoscopic and stereoscopic (active, passive and anaglyphic) rendering.
\item Generic, transparent head tracking in the core framework, based on
  OpenCV. The tracking will provide X,Y translation, an estimation of the
  observer's Z distance, and when possible, roll estimation using the eye
  positions.
\item Object manipulation using the Logitech 3D Spacemouse. The Spacemouse
  delivers relative sensor data for six degrees of freedom (rotation and
  translation), which will be used to move an object in the scene accordingly.
\item create a waterfall modeled using a particle system based on a physics
  library, which will flow around the displayed model. A spatial data structure
  is used to ply model the interaction between the model and the particle
  system.
\end{itemize}

\fix{screenshots}

\section{Tools and Methods}

\fix{How the 3D scene was built and which tools were used. Libraries used for
  the head tracking, fiducial marker tracking, physics simulation and any
  additional feature.}

\section{Installation}
\fix{Share your experience. What were the difficulties? What you liked and/or
  disliked while developing this project? What you have learned?}

\section{Conclusion}

\end{document}
