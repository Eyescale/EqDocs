\NeedsTeXFormat{LaTeX2e}
\documentclass[10pt,a4]{scrartcl}

\usepackage{tabularx,longtable,graphicx,a4,listings,wrapfig,subfigure,textcomp,ccaption,latexsym,setspace,algorithm,algorithmic,esvect,color}
\usepackage[absolute]{textpos}

\ifx\pdfoutput\undefined
  % We're not running pdftex
  % european (better) fonts -- does not look good with pdflatex
  \usepackage[T1]{fontenc}
  \newcommand{\href}[2]{#2\\{\hspace*{5mm}\scriptsize <#1>}\\}
\else
  \pdfcompresslevel=9
  \def\pdfBorderAttrs{/Border [0 0 0] } % No border around Links
  \usepackage{hyperref}
\fi

\title{Project, Virtual Reality}
\author{Stefan Eilemann, Benjamin Long}

\newcommand{\tm}{\texttrademark~}
\newcommand{\rc}{\raise 1ex\hbox{{\tiny\textregistered}}~}
\newcommand{\fig}[1]{Figure~\ref{#1}}
\newcommand{\sref}[1]{Section~\ref{#1}}
\newcommand{\aref}[1]{Appendix~\ref{#1}}
\newcommand{\link}[1]{\htmladdnormallink{#1}{#1}}
\newcommand{\fix}[1]{\textbf{\color{red}{#1}}}
\indentcaption{2em}

% suppress  single floating lines on top (widow) and bottom(club)
%  10000 is infinity
%  tradeoff: maybe underfull vboxes
\clubpenalty=10000
\widowpenalty=10000

\begin{document}
\maketitle

\section{Project Description}

The project builds upon the Equalizer parallel rendering framework and one of
its example programs, eqPly. It adds the following features to either the
core framework, where it is generic, or to the application, where it is not:
\begin{itemize}
\item Render any 3D object in the ply file format on any display system
  supported by Equalizer (LCD, stereo screen, CAVE, display wall) using
  monoscopic and stereoscopic (active, passive and anaglyphic) rendering.
\item Generic, transparent head tracking in the core framework, based on
  OpenCV. The tracking will provide X,Y translation, an estimation of the
  observer's Z distance, and when possible, roll estimation using the eye
  positions.
\item Object manipulation using the Logitech 3D Spacemouse. The Spacemouse
  delivers relative sensor data for six degrees of freedom (rotation and
  translation), which will be used to move an object in the scene accordingly.
\item create a waterfall modeled using a particle system based on a physics
  library, which will flow around the displayed model. A spatial data structure
  is used to ply model the interaction between the model and the particle
  system.
\end{itemize}

\fix{screenshots}

\section{Tools and Methods}

The project is integrated into the Equalizer parallel rendering framework, where
possible in the core library. The non-generic parts of the work are build into
\textsf{eqPly}, the polygonal rendering example shipped with Equalizer. The
Equalizer programming and user guide has a detailed description on the structure
of \textsf{eqPly}

OpenCV is used for head tracking. The support of OpenCV is generic in Equalizer,
a new field configures the camera index used for each observer. The support is
conditional on the availability of the OpenCV library and headers at compile
time, as well as on the presence of a camera at runtime. A face and eye
detection filter continuously estimates the 3D position in space, and if eyes
are detected, the roll of the observer. Head and pitch are not estimated due to
the limited amount of freedom given by a single camera input. Since the computer
vision is computationally expensive, the OpenCV head tracker runs asynchronous
in a separate thread from the main application loop. After each detection, it
injects an observer event into Equalizer, which is processed by the normal event
handling flow.

For interaction, a 3D space mouse is used. This device reports relative values
for six degrees of freedom. Again, the spacemouse support is fully integrated
into the Equalizer event handling code, and the events are processed by
\textsf{eqPly} to manipulate the model matrix applying the six delta values in a
one-to-one mapping onto the model. Three different backend implementations
receive and convert the event from the device into a generic representation: a
\textsf{spnav} implementation for GLX (X11, Linux), an implementation based on
the official driver for AGL (Carbon, Mac OS X) and a raw message-based
implemenation for WGL (Windows).

\fix{Physics:}

\section{Installation}

\section{Discussion}

This project contributed new functionality into an existing open source
project. While this added requirements to the project, this work immediately
benefits the existing users. Furthermore, building unto an existing framework
also facilitates the implementation due to the reuse of existing infrastructure
and functionality, e.g., the already-build rendering application. The resulting
demo is also rich in features and can easily be deployed on virtually all types
of display systems.

Due to the previous knowledge with Equalizer, the integration of the spacemouse
and OpenCV was relatively straightforward. The correct design of integrating
camera-based tracking took some time to figure o out. At first, a synchronous
approach polling the camera each frame was used, where the actual detection was
used to another thread. This proved too complex and hard to integrate cleanly
into the existing design, which was finally solved by an asynchronous approach
injecting events into the already distributed event processing. The chosen
design is simple in its implementation and robust in execution.

\fix{Share your experience. What were the difficulties? What you liked and/or
  disliked while developing this project? What you have learned?}

\end{document}
