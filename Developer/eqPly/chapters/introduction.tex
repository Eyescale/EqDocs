% Copyright (c) 2007, Tobias Wolf <twolf@access.unizh.ch>
% All rights reserved.  

%
% Kapiteldatei
%


\chapter{Einleitung}

Die Leistung von Grafikhardware hat sich in den letzten Jahren kontinuierlich 
und rasant weiter entwickelt. Mit h�herer Leistungsf�higkeit gehen aber auch 
h�here Anspr�che einher. Es besteht der Wunsch, immer komplexer werdende Daten 
mit steigendem Datenvolumen in h�heren Aufl�sungen zu visualisieren. Trotz des 
Fortschrittes gelangen tradionelle Computergrafik Systeme in gewissen 
Anwendungsgebieten an ihre Grenzen, denn obwohl sich die Hardwareleistung 
in regelm�ssigen Abst�nden verdoppelt, wachsen die produzierten und 
darzustellenden Datenmengen mindestens genau so schnell 
\cite{pajarola:datasize}.

Dem �hnlich gelagerten Problem einer zu geringen Rechenleistung einzelner 
Computersysteme wurde begegnet, indem mehrere verbundene Rechnersysteme die
gew�nschten Berechnungen verteilt aber parallel ausf�hren. Analog dazu l�sst
sich auch im Bereich der Computergrafik die notwendige Arbeit auf mehrere
Grafiksysteme verteilen, welche dann einzelne Teilbereiche parallel bearbeiten 
und die Teilergebnisse wo notwendig wieder zu einem Gesamtbild zusammensetzen.

\section{Aufgabenstellung}

\emph{Equalizer} \cite{equalizer:tr}\cite{equalizer:main} ist ein 
L�sungsansatz f�r die im letzten Abschnitt vorgestellte Problematik. Es handelt 
sich bei Equalizer um ein Open Source Framework f�r verteiltes, paralleles 
\emph{Rendering}\footnote{Englische Begriffe werden im Glossar am Ende des
Dokumentes �bersetzt oder erkl�rt.}, welches sich zur Zeit in Entwicklung 
befindet. Es setzt auf \emph{OpenGL} \cite{opengl:main} auf, der bedeutendsten 
Programmierschnittstelle f�r plattform�bergreifende, anspruchsvolle 
Grafikprogrammierung.

\glossary{name={Rendering},description={Erzeugung eines digitalen Bildes aus 
einer Bildbeschreibung.}}

Ziel der vorliegenden Semesterarbeit ist die Entwicklung einer Applikation f�r 
die effiziente und optisch ansprechende Darstellung von polygonalen Daten, auf
Basis des Equalizer Frameworks und der OpenGL Programmierschnittstelle. Ein 
bereits existierendes Beispielprogramm wird als Grundlage verwendet, und soll 
am Ende durch das Ergebnis dieser Arbeit vollst�ndig ersetzt werden.

Die erste Teilaufgabe besteht darin, eine effiziente Datenstruktur f�r die 
Verwaltung von polygonalen Daten umzusetzen. Die Wahl fiel hierbei auf eine 
besondere Baumstruktur, den sogenannten \emph{kd-Tree} \cite{kd-tree:desc}.

\glossary{name={kd-Tree},description={K-dimensionaler Baum, eine Datenstruktur
aus der Informatik zur r�umlichen Aufteilung.}}

Die polygonalen Daten in Form von \emph{Vertices} und \emph{Triangles} werden 
aus Dateien im PLY Dateiformat \cite{ply:desc} gelesen, und anschliessend in 
die Baumstruktur einsortiert. Die fertige Datenstruktur erlaubt sowohl ein 
einfaches \emph{View Frustum Culling} als auch ein effizientes Rendering. Da 
das Erstellen des kd-Trees ein sehr aufw�ndiger Prozess ist, werden die Daten 
f�r zuk�nftige Verwendung in einem eigenen Bin�rformat zwischengespeichert.

\glossary{name={Vertices},description={Eckpunkte grafischer Objekte.}}
\glossary{name={Triangles},description={Dreiecke. Elementare Objekte der
Computergrafik aus denen komplexere Objekte typischerweise zusammengesetzt
sind.}}
\glossary{name={View Frustum Culling},description={Eliminierung von 
Objekten die ausserhalb des sichtbaren Bereiches liegen vor dem Rendering.}}

Die zweite Teilaufgabe schliesslich ist die Implementierung eines Rendering 
Verfahrens f�r die optisch ansprechende Darstellung der polygonalen Daten. 

Die Datenstruktur ist optimiert f�r die Verwendung von \emph{Vertex Buffer 
Objects} (VBOs) \cite{vbo:whitepaper}, welche entsprechend auch f�r das
Rendering verwendet werden sollen. Zur Verbesserung der visuellen Qualit�t
sollen verschiedene Algorithmen einsetzt werden, welche mit Hilfe der 
\emph{Open GL Shader Language} (GLSL) \cite{opengl:glsl} entwickelt werden 
und direkt auf der Grafikhardware ablaufen. Damit auch �ltere OpenGL 
Umgebungen, welche weder VBOs (ab Version 1.5) noch GLSL (ab Version 2.0) 
unterst�tzen, die Applikation ausf�hren und darstellen k�nnen, wird eine 
OpenGL Version 1.1 kompatible L�sung ebenfalls implementiert.

\section{Gliederung}

Der Aufbau der Arbeit sieht wie folgt aus. Nach der Einf�hrung in diesem
Kapitel besch�ftigt sich Kapitel \ref{application} mit dem grundliegenden
Aufbau der Applikation. Weil dazu auch ein gewisses Verst�ndnis der Equalizer
Architektur notwendig ist, werden wichtige Elemente daraus kurz vorgestellt.

Kapitel \ref{data-structure} widmet sich ausf�hrlich dem ersten Teil der 
Aufgabe. Es wird die bisher verwendete Datenstruktur erkl�rt und deren
Schw�chen aufgezeigt. Die neue Datenstruktur wird vorgestellt und auch auf
deren konkrekte Implementierung eingegangen.

Der zweite Teil der Aufgabe wird in Kapitel \ref{rendering} behandelt. Es
werden zun�chst die verschiedenen m�glichen OpenGL Rendermodi vorgestellt, 
und dann deren Verwendung in der bestehenden und neuen Applikation aufgezeigt.
Anschliessend werden die unterschiedlichen Algorithmen des \emph{Shading} 
diskutiert, sowie deren Anwendung sowohl in Standard OpenGL als auch mit GLSL.

\glossary{name={Shading},description={Simulation der Oberfl�chen grafischer
Objekte anhand der Beleuchtung und Materialeigenschaften.}}

Den Abschluss der Arbeit stellt Kapitel \ref{conclusion} dar, in welchem 
das Ergebnis der Arbeit kurz zusammengefasst sowie ein Ausblick auf weitere
zuk�nftige Erweiterungen der Applikation geworfen wird.


%
% EOF
%
