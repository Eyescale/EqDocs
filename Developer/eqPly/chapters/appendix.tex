% Copyright (c) 2007, Tobias Wolf <twolf@access.unizh.ch>
% All rights reserved.  

%
% Kapiteldatei
%


\appendix
\chapter{Inhalt der CD-ROM}

Die der Semesterarbeit beiliegende CD-ROM besitzt den folgenden Inhalt:

\begin{table}[hb]
\centering
\begin{tabular}{ll}
\textbf{Pfad} & \textbf{Beschreibung} \\
zusfg.pdf & Zusammenfassung der Arbeit in Deutsch \\
abstract.pdf & Zusammenfassung der Arbeit in Englisch \\
semester.pdf & Elektronische Version der Arbeit \\
latex & LaTeX Quellen der Arbeit \\
eq & Equalizer SVN Snapshot per 30. September 2007 \\
\end{tabular}
\caption{Inhalt der CD-ROM}
\label{tbl:cd-contents}
\end{table}

Obwohl die CD-ROM einen kompletten Equalizer SVN Snapshot enth�lt, wird
dringend empfohlen bei Bedarf eine aktuelle Version aus dem Repository
auszuchecken \cite{equalizer:downloads}. Sowohl Equalizer als auch eqPly 
befinden sich noch in aktiver Entwicklung.

Der Source Code der urspr�nglichen Beispielapplikation befindet sich im
Verzeichnis \emph{eq/src/examples/eqPly/}. Das Ergebnis dieser Arbeit, der
Source Code der neuen Applikation, liegt in \emph{eq/src/examples/eqPlyNew/}.

Um die Applikationen auszuf�hren ist ein vollst�ndiges Kompilieren von 
Equalizer notwendig. Unter Mac OS X und Linux reicht es aus, dazu in das 
\emph{eq/src} Verzeichnis zu wechseln und \emph{make} auszuf�hren. F�r 
Windows enth�lt \emph{eq/src/VS2005} eine Visual Studio 2005 Solution. 
Die notwendigen Voraussetzungen und Hinweise f�r das Kompilieren unter 
den verschiedenen Betriebssystemen finden sich in den \emph{README} Dateien
sowie der \emph{FAQ} im Verzeichnis \emph{eq/src/}.


%
% EOF
%
